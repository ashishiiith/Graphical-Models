% !TEX TS-program = pdflatex
% !TEX encoding = UTF-8 Unicode

% This is a simple template for a LaTeX document using the "article" class.
% See "book", "report", "letter" for other types of document.

\documentclass[11pt]{article} % use larger type; default would be 10pt

\usepackage[utf8]{inputenc} % set input encoding (not needed with XeLaTeX)
\usepackage{amsmath,amsthm, amssymb}

%%% Examples of Article customizations
% These packages are optional, depending whether you want the features they provide.
% See the LaTeX Companion or other references for full information.

%%% PAGE DIMENSIONS
\usepackage[margin=0.4in]{geometry} % to change the page dimensions
\geometry{a4paper} % or letterpaper (US) or a5paper or....
% \geometry{margin=2in} % for example, change the margins to 2 inches all round
% \geometry{landscape} % set up the page for landscape
%   read geometry.pdf for detailed page layout information

\usepackage{graphicx} % support the \includegraphics command and options
\usepackage[makeroom]{cancel}
\renewcommand{\familydefault}{\sfdefault}
% \usepackage[parfill]{parskip} % Activate to begin paragraphs with an empty line rather than an indent

%%% PACKAGES
\usepackage{booktabs} % for much better looking tables
\usepackage{array} % for better arrays (eg matrices) in maths
\usepackage{paralist} % very flexible & customisable lists (eg. enumerate/itemize, etc.)
\usepackage{verbatim} % adds environment for commenting out blocks of text & for better verbatim
\usepackage{subfig} % make it possible to include more than one captioned figure/table in a single float
% These packages are all incorporated in the memoir class to one degree or another...

%%% HEADERS & FOOTERS
\usepackage{fancyhdr} % This should be set AFTER setting up the page geometry
\pagestyle{fancy} % options: empty , plain , fancy
\renewcommand{\headrulewidth}{0pt} % customise the layout...
\lhead{}\chead{}\rhead{}
\lfoot{}\cfoot{\thepage}\rfoot{}

%%% SECTION TITLE APPEARANCE
\usepackage{sectsty}
\allsectionsfont{\sffamily\mdseries\upshape} % (See the fntguide.pdf for font help)
% (This matches ConTeXt defaults)

%%% ToC (table of contents) APPEARANCE
\usepackage[nottoc,notlof,notlot]{tocbibind} % Put the bibliography in the ToC
\usepackage[titles,subfigure]{tocloft} % Alter the style of the Table of Contents
\renewcommand{\cftsecfont}{\rmfamily\mdseries\upshape}
\renewcommand{\cftsecpagefont}{\rmfamily\mdseries\upshape} % No bold!

%%% END Article customizations

%%% The "real" document content comes below...

\title{Assignment-1}
\author{Ashish Jain}
%\date{} % Activate to display a given date or no date (if empty),
         % otherwise the current date is printed 

\begin{document}
\maketitle

\section*{Question 1. Factorization}
{\raggedleft{}Joint distribution for any Bayesian Network is obtained as a product of these conditional distribution:}
\begin{align*}
P(X) = \prod_{i=1}^{N} P(X_i | Pa_{X_i}^{G}) \\
\end{align*}
Using the above equation, we can write the factorization for the given graph as follows:
\begin{align*}
P(A,G,BP,CH,HD,CP,EIA,ECG,HR) &= P(G)P(BP|G)P(CH|G, A)P(HD|BP, CH)P(HR|A, HD)\\
								&P(CP|HD)P(EIA|HD)P(ECG|HD)
\end{align*}
\section*{Question 2. Likelihood Function}

Log likelihood function as an empirical average over the data set is given by following expression:

\begin{equation}
{\cal L}(\theta) = \dfrac{1}{N} \sum_{n=1}^{N} log P_{\theta}(x_{n}) \\
\end{equation}
\begin{equation}
P_{\theta}(X=x) = \prod_{d=1}^{D} P_{\theta}(X_{d} = x_{d}|X_{Pa(X_{d})} = x_{Pa(X_{d})}) \\
				= \prod_{d=1}^{D}\prod_{v=1}^{V}(\theta_{v|x_{Pa(X_{d})}}^{X_{d}})^{[x_{d}=v]}
\end{equation}
Using the above two equations we will write the probability expression for the given graph and than take log of it.
\begin{flalign*}
P_{\theta}(A,G,BP,CH,HD,CP,EIA,ECG,HR) &= P_{\theta}(G=g)P_{\theta}(BP=bp|G=g)\\
& P_{\theta}(CH=ch|G=g, A=a)P_{\theta}(HD=hd|BP=bp, CH=ch)\\
& P_{\theta}(HR=hr|A=a,HD=hd)P_{\theta}(CP=cp|HD=hd)\\
& P_{\theta}(EIA=eia|HD=hd)P_{\theta}(ECG=ecg|HD=hd) \\ \\
\end{flalign*}
\begin{flalign*}
{\cal L}(\theta) &= \dfrac{1}{N} \sum_{n=1}^{N} log P_{\theta}(x_{n}) \\ 
&= \dfrac{1}{N} \sum_{n=1}^{N} \sum_{g}[g_{n}=g]log P(G=g) + \dfrac{1}{N} \sum_{n=1}^{N} \sum_{a}[a_{n}=a]log P(A=a) \\
& + \dfrac{1}{N} \sum_{n=1}^{N} \sum_{bp,g}[bp_{n}=bp][g_{n}=g]log P(BP=bp|G=g) \\
& + \dfrac{1}{N} \sum_{n=1}^{N} \sum_{ch,a,g}[ch_{n}=ch][g_{n}=g][a_{n}=a]log P(CH=ch|G=g, A=a) \\
& + \dfrac{1}{N} \sum_{n=1}^{N} \sum_{hd,bp,ch}[hd_{n}=hd][bp_{n}=bp][ch_{n}=ch]log P(HD=hd|BP=bp,CH=ch) \\
& + \dfrac{1}{N} \sum_{n=1}^{N} \sum_{hr,a,hd}[hr_{n}=hr][a_{n}=a][hd_{n}=hd]log P(HR=hr|A=a,HD=hd) \\
& + \dfrac{1}{N} \sum_{n=1}^{N} \sum_{cp, hd}[cp_{n}=cp][hd_{n}=hd]log P(CP=cp|HD=hd) \\ 
& + \dfrac{1}{N} \sum_{n=1}^{N} \sum_{eia, hd}[eia_{n}=eia][hd_{n}=hd]log P(EIA=eia|HD=hd) \\
& + \dfrac{1}{N} \sum_{n=1}^{N} \sum_{ecg, hd}[ecg_{n}=ecg][hd_{n}=hd]log P(ECG=ecg|HD=hd) \\
\end{flalign*}

\begin{flalign*}
{\cal L}(\theta) &= \dfrac{1}{N} \sum_{n=1}^{N} \sum_{g}[g_{n}=g]log \theta_{a}^{A} + \dfrac{1}{N} \sum_{n=1}^{N} \sum_{a}[a_{n}=a]log \theta_{g}^{G} \\
& + \dfrac{1}{N} \sum_{n=1}^{N} \sum_{bp,g}[bp_{n}=bp][g_{n}=g]log \theta_{bp|g}^{BP} \\
& + \dfrac{1}{N} \sum_{n=1}^{N} \sum_{ch,a,g}[ch_{n}=ch][g_{n}=g][a_{n}=a]log \theta_{ch|g,a}^{CH} \\
& + \dfrac{1}{N} \sum_{n=1}^{N} \sum_{hd,bp,ch}[hd_{n}=hd][bp_{n}=bp][ch_{n}=ch]log \theta_{hd|ch,bp}^{HD} \\
& + \dfrac{1}{N} \sum_{n=1}^{N} \sum_{hr,a,hd}[hr_{n}=hr][a_{n}=a][hd_{n}=hd]log \theta_{hr|a,hd}^{HR} \\
& + \dfrac{1}{N} \sum_{n=1}^{N} \sum_{cp, hd}[cp_{n}=cp][hd_{n}=hd]log \theta_{cp|hd}^{CP} \\ 
& + \dfrac{1}{N} \sum_{n=1}^{N} \sum_{eia, hd}[eia_{n}=eia][hd_{n}=hd]log \theta_{eia|hd}^{EIA} \\
& + \dfrac{1}{N} \sum_{n=1}^{N} \sum_{ecg, hd}[ecg_{n}=ecg][hd_{n}=hd]log \theta_{ecg|hd}^{ECG} \\
\end{flalign*}
\section*{Question 4. Learning}
\begin{tabular}{|c|c|}
\hline
P(A)   & (A)   \\ \hline
0.1769 & \textless45   \\ \hline
0.3086 & 45-55 \\ \hline
0.5144 & \textgreater=55  \\ \hline
\end{tabular}
\vspace*{1em}\\
\begin{tabular}{|c|c|c|}
\hline
P(BP$|$G) & P(BP) & P(G)   \\ \hline
0.3658  & Low   & Female \\ \hline
0.6341  & High  & Female \\ \hline
0.472   & Low   & Male   \\ \hline
0.5279  & High  & Male   \\ \hline
\end{tabular}
\vspace*{1em}\\
\begin{tabular}{|c|c|c|c|}
\hline
P(HD$|$BP, CH) & HD & BP   & CH   \\ \hline
0.5263         & N  & Low  & Low  \\ \hline
0.4736         & Y  & Low  & Low  \\ \hline
0.5909         & N  & High & Low  \\ \hline
0.409          & Y  & High & Low  \\ \hline
0.5862         & N  & Low  & High \\ \hline
0.4137         & Y  & Low  & High \\ \hline
0.513          & N  & High & High \\ \hline
0.4869         & Y  & High & High \\ \hline
\end{tabular}
\vspace*{1em} \\ \\
\begin{tabular}{|c|c|c|c|}
\hline
P(HR$|$A, HD) & HR   & A      & HD \\ \hline
0.0606        & Low  & $<$45  & N  \\ \hline
0.9393        & High & $<$45  & N  \\ \hline
0.6           & Low  & 45-55 & N  \\ \hline
0.4           & High & 45-55 & N  \\ \hline
0.173         & Low  & $>=$55 & N  \\ \hline
0.8269        & High & $>=$55 & N  \\ \hline
0.5217        & Low  & $<$45 & Y  \\ \hline
0.4782        & High & $<$45 & Y  \\ \hline
0.3333        & Low  & 45-55 & Y  \\ \hline
0.6666        & High & 45-55 & Y  \\ \hline
0.5714        & Low  & $>=$55 & Y  \\ \hline
0.4285        & High & $>=$55 & Y  \\ \hline
\end{tabular}
\vspace*{1em}\\
\section*{5. Probability Queries}
We will use following joint probability expression for the given two queries:
\begin{align*}
P(A,B) &= P(A|B).P(B)
\end{align*}
\subsection*{(a)}
Random variable CH (cholestrol) can take following two values: Low and High. Let us solve the query using $CH=L$ using the above joint probability equation.
 \begin{flalign*}
  P(CH = L | A = 2, G = M, CP = None, BP = L, ECG = Normal, HR = L, EIA = No, HD = No) = \\
  \dfrac{P(CH = L, A = 2, G = M, G = M, CP = None, BP = L, ECG = Normal, HR = L, EIA = No, HD = No)}
          {P(A = 2, G = M, CP = None, BP = L, ECG = Normal, HR = L, EIA = No, HD = No)} =   \\
   \dfrac{P(CH = L, A = 2, G = M, G = M, CP = None, BP = L, ECG = Normal, HR = L, EIA = no, HD = no) }
          {\sum_{ch \in (L, H) }P(CH = ch, A = 2, G = M, CP = None, BP = L, ECG = Normal, HR = L, EIA = no, HD = no} = \tag{\text{marginalizing over CH} }\\
   \dfrac{P(CH = L | A = 2, G = M)P(HD = L | CH = L, BP = L)}{\sum_{ch \in (L, H)} P(CH = ch | A = 2, G = M) P(HD = L| CH = ch, BP = L)} 
      = \tag{\text{Using factorization and conditional independence property, terms indepedent of CH will get cancelled out.} } \\
 \end{flalign*} \\
 By using learned CPT tables in Part 4 over training file 1, we get following answer for this Query: \\
 $P(CH = L | A = 2, G = M, CP = None, BP = L, ECG = Normal, HR = L, EIA = No, HD = No) = $0.1522  \\
 $P(CH = R | A = 2, G = M, CP = None, BP = L, ECG = Normal, HR = L, EIA = No, HD = No) = $0.8477  \\
\subsection*{(b)}
BP can take two values: Low and High. Let us solve the expression for $BP=L$. We have unobserved variable $G$ in this Query. \\
\begin{gather*}
  P(BP =L | A = 2, CP = Typical, CH = H, ECG = Normal, HR = H, EIA = Yes, HD = No) =\\  
  \dfrac{P(BP = L, A = 2, CP = Typical, CH = H, ECG = Normal, HR = H, EIA = Yes, HD = No)}
          {\sum_{bp} P(BP = bp, A = 2, CP = Typical, CH = H, ECG = Normal, HR = H, EIA = Yes, HD = No)} = \tag{\text{marginalizing over $BP$ in denominator} }\\ 
  \dfrac{\sum_{g} P(BP = L, A = 2, G = g, CP = Typical, CH = H, ECG = Normal, HR = H, EIA = Yes, HD = No)}
    {\sum_{bp} \sum_{g} P(BP = bp, A = 2, G = g, CP = Typical, CH = H, ECG = Normal, HR = H, EIA = Yes, HD = No)}
    \tag{\text{marginalizing over unobserved variable $G$ in denominator and numerator.} }\\ 
\end{gather*}\\
Finally we get after applying factorization and canceling out the terms in numerator and denominator  :\\
{\tiny{
\begin{gather*}
\dfrac{\sum_{g} P(G=g) P(CH=H|G=g, A=2) P(BP=L|G=g) P(HR=H|A=2, BP=L, HD=No)P(HD=No| BP=L, CH=H)}
    {\sum_{bp} \sum_{g} P(G=g) P(CH=H | G=g, A=2) P(BP=bp | G=g) P(HR=H| A=2, BP=bp, HD=No)P(HD=No| BP=bp, CH=H)}
\end{gather*}\\
}}
Using the CPT tables learned on training file 1 in Question4, we get \\ \\
$P(BP =L | A = 2, CP = Typical, CH = H, ECG = Normal, HR = H, EIA = Yes, HD = No) =$ 0.4685 \\
$P(BP =R | A = 2, CP = Typical, CH = H, ECG = Normal, HR = H, EIA = Yes, HD = No) =$ 0.5314 \\

\section*{6. Classification}
\subsection*{(b)}

\subsection*{(c)}
\subsection*{Part (c) : }

\begin{tabular}{|c|c|c|c|}\hline
Fold & Correct & Total & $Accuracy $  \\ \hline
1 & 44 & 60 & 73.33 \\ \hline
2 & 48 & 60 & 80 \\ \hline
3 & 40 & 60 & 66.66 \\ \hline
4 & 48 & 60 & 80 \\ \hline
5 & 47 & 60 & 78.33\\ \hline
Mean & 45.4 & 60 & 75.66 \\\hline
\end{tabular}
\vspace{1em} \\
Mean Prediction accuracy over the five test files $=75.66$. \\
Standard deviation of the prediction accuracy over the five test files $=5.12$.
\end{document}
